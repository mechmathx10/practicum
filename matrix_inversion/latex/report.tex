\documentclass{article}
\usepackage[russian]{babel}
\usepackage[utf8]{inputenc}
\usepackage{dsfont}
\usepackage{natbib}
\usepackage{graphicx}
\usepackage{amsmath}

\title{Нахождение обратной матрицы блочным методом Жордана-Гаусса с выбором главного элемента по строке.}
\author{Тырин Владимир, группа 310}
\date{}

\begin{document}

\maketitle

\section{Постановка задачи}
Дана вещественная квадратная матрица $A$. Найти обратную матрица $A^{-1}$ методом Жордана-Гаусса с выбором главного элемента по строке.

\section{Описание стандартного метода}

Опишем стандартный (не блочный) метод Жордана-Гаусса с выбором главного элемента по строке.
Пусть $A$ --- квадратная матрица порядка $n$. Припишем к ней справа единичную матрицу. Получим в результате матрицу порядка $n\times 2n$ обозначим через $(A_0|B_0) = (A|E)$, а впоследствии получающиеся из нее после матрицы через $(A_i|B_i)$. Элементы матриц в дальнейшем обозначаются через $a_{ikl}$, где $i$ --- номер шага алгоритма, $k$ --- номер строки, $l$ --- номер столбца.
\[ 
(A_0|B_0) = \left( 
\begin{array} {rrrrr|rrrrr}
a_{11} & a_{12} & \ldots & a_{1,n-1} & a_{1n} & 1 & 0 & \ldots & 0 & 0 \\ 
a_{21} & a_{22} & \ldots & a_{2,n-1} & a_{2n} & 0 & 1 & \ldots & 0 & 0 \\ 
\ldots & \ldots & \ldots & \ldots & \ldots & \ldots & \ldots & \ldots & \ldots & \ldots \\ 
a_{n-1,1} & a_{n-1,2} & \ldots & a_{n-1,n-1} & a_{n-1,n} & 0 & 0 & \ldots & 1 & 0 \\
a_{n1} & a_{n2} & \ldots & a_{n,n-1} & a_{nn} & 0 & 0 & \ldots & 0 & 1 \\ 
\end{array} 
\right) 
\]

В алгоритме также будут использоваться перестановки столбцов матрицы. Для того, чтобы из результата преобразований корректно получить обратную матрицу, эти перестановки нужно запоминать. Начальная перестановка --- тождественная.
$$ \sigma_0 = \textbf{id} \in S_n $$
Все преобразования, описанные далее, выполняются одновременно для матриц $A_i$ и $B_i$. (перестановки столбцов, элементарные преобразования строк).
Алгоритм содержит $n$ шагов, $i$-ый из которых ($i=1..n$) устроен следующим образом:
\begin{enumerate}

\item Если $i$-ая строка --- нулевая, алгоритм завершается: матрица не имеет обратной. В противном случае выберем в $i$-ой строке максимальный по модулю элемент. Обозначим номер содержащего его столбца через $j_i$. Переставим в матрицах $A_{i-1}$ и $B_{i-1}$ столбцы с номерами $i$ и $j_i$, после чего разделим $i$-ые строки полученных матриц на $a_{i-1,i,j_i}$. Результат обозначим через $(\tilde{A_i}|\tilde{B_i})$.

\item Домножим перестановку $\sigma_{i-1}$ слева на транспозицию $(i,j_i)$:
$$ \sigma_i = (i,j_i) \cdot \sigma_{i,j_i} $$

\item Вычтем из всех последующих строк (номер уменьшаемой строки обозначим через $k,\,k\in i+1..n$) $i$-ую строку, домноженную на $\tilde{a}_{i-1,li}$. Результат выполнения операции обозначим через $(A_i|B_i)$. Выполнены соотношения для элементов:
$$ a_{ikl} = \tilde{a}_{i-1,kl},\quad k=1..i $$
$$ a_{ikl} = \tilde{a}_{i-1,kl} - \tilde{a}_{i-1,il}\tilde{a}_{i-1,ki},\quad l=i+1..n $$
$$ b_{ikl} = \tilde{b}_{i-1,kl},\quad k=1..i $$
$$ b_{ikl} = \tilde{b}_{i-1,kl} - \tilde{b}_{i-1,il}\tilde{b}_{i-1,ki},\quad l=i+1..n $$ 

\end{enumerate}

После выполнения шага $i.3$ ($x.y$ --- $y$-ый этап $x$-го шага алгоритма), очевидно, столбцы матрицы $A_i$ с номерами $1..i$ совпадают с сооьветствующими столбцами единичной матрицы порядка $n$. Соответственно, $$A_n = E$$

\end{document}